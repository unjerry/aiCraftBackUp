\documentclass{book}

\title{计算线代}
\newcommand{\booksubtitle}{面向计算的线性代数}
\newcommand{\booklicense}{ALL RIGHT PRESERVED}

\author{张津睿}
% Author subtitle could be a university or a geographical location, for example
\newcommand{\authorsubtitle}{新域, 中国}

% Create convenient commands \booktitle and \bookauthor
\makeatletter
\newcommand{\booktitle}{\@title}
\newcommand{\bookauthor}{\@author}
\makeatother

% This utf8 declaration is not needed for versions of latex > 2018 but may
% be helpful for older software. Eventually it may not be worth keeping.
\usepackage[utf8]{inputenc}  
\usepackage{fix-cm}  % this package allows large \fontsize
\usepackage{tikz}    % this is for graphics. e.g. rectangle on title page
\usepackage{amsmath} % Used by equations
\usepackage{amssymb}
\usepackage{CJKutf8}
\usepackage[UTF8]{ctex}
\usepackage[center]{titlesec}%chapter1修改为第1章
\usepackage{hyperref}%生成pdf超链接目录
\hypersetup{hidelinks,
	colorlinks=true,
	allcolors=black,
	pdfstartview=Fit,
	breaklinks=true
}
\usepackage{url}
% \usepackage[colorlinks,linkcolor=blue,anchorcolor=blue,citecolor=blue]{hyperref}
\titleformat{\chapter}{\raggedright\Huge\bfseries}{第\,\thechapter\,章}{1em}{}
\titleformat{\section}{\raggedright\Large\bfseries}{\,\thesection\,}{1em}{}
\titleformat{\subsection}{\raggedright\large\bfseries}{\,\thesubsection\,}{1em}{}

% The following dimensions specify 4.75" X 7.5" content on 6 3/8" by 9 1/4"
% paper. The paper width and height can be tweaked as required and the content
% should size to fit within the margins accordingly.
%
% The (inside) bindingoffset should be larger for books with more pages. Some
% standard recommended sizes are .375in minimum up to 1in for 600+ page books.
% Sizes .75in and .875in are also recommended roughly at 150 and 400 pages.
\usepackage[bindingoffset=0.625in,
            left=.5in, right=.5in,
            top=.8125in, bottom=.9375in,
            paperwidth=6.375in, paperheight=9.25in]{geometry}
% Here is an alternative geometry for reading on letter size paper:
% \usepackage[margin=.75in, paperwidth=8.5in, paperheight=11in]{geometry}

\renewcommand{\contentsname}{Table of Contents} % default is {Contents}
\usepackage{makeidx}
\makeindex % Initialize an index so we can add entries with \index

% The next few commands are for creating fake content to fill out the template.
% You should delete this (e.g.  everything up to, but not including,
% \begin{document}) after you insert your own content.
% Example content from Einstein's Meaning of Relativity.
% Public domain book: http://www.gutenberg.org/ebooks/36276
\newcommand{\fakeparagraph}{}
\newcommand{\para}{defsdf}
\newcommand{\fakecontent}{
    sldkfjsldkfj

    sdfsdfsdf
}
\newcommand{\prefacePara}
{
现今市面上的线性代数教材非常充足,像我这样一个无
名小卒来写一本教材必然会贻笑大方。但是我并不能放
下那些我有时候自己脑海中的古怪想法和天马行空的路
线。我是看面向算子的线性代数入门的,我读的是
Axler, Sheldon写的\cite[Linear algebra done right]{1664},
我也感觉非常庆幸能够在学生时代读到那样的好书。

本书将基于一种面向计算的方式来讲解一般意义上的
线性代数。线性代数,线性是一个非常广阔的话题,我
非常惶恐地尝试将我在这片非常精彩的领域所学到的东
西分享在这本书中。

这本书会回答所有我曾经感到非常困惑和未能够得到任
意一位老师满意答案的问题。包括行列式为什么如此定
义;行向量和列向量到底有什么区别;转置究竟在干什么
,为什么转置有和逆一样的性质(乘积的转置等于分别
转置但是交换相乘);迹又是什么,为什么单单把对角
线元素加起来就得到迹,这种搞笑的计算方法为什么有
那么多神奇的性质,都是那里来的?

要搞清楚这些事情并不容易,但是在当下这个计算机技术发达的时代,
其实你会有更多的选择,实际上如果你想要搞明白已经存在的数学知识,
你其实只需要一个手机和一个有自由知识环境的网络。包括一些非常棒的
网络课程系列,比如\cite[线性代数的本质]{citee}。我非常建议看完
这个课程再来看这本书。如果你看完课程发现完全搞懂了所有事情,没有任何疑问
你其实可以不用看这本书了。因为这本书就是在我看完那个课程的很多年
,当我解决了所有当年的疑问(原谅我的愚钝,我实在有太多疑问和不满意的地方)
和未被解答的问题以后才决定写的。为了寻求一个满意的答案,
我实在是看来太多视频课程,本书也可以看作我的一个学习笔记,
如果你看到书中的某些例子或者理解,早在一个你知道的课程中出现,那你直接去看那个课程就行了。
大概率我只是忘了我看过那个课程所以它没有出现再本书的引用当中。

本书主要会有三部分内容,全部以大量例子杂糅在一起。
分别是张量相关的话题,外代数(几何代数)和传统线性
代数。相信我,前两个话题对于理解线性代数里面一些看
起来“凭空”产生的命题和定理有很大帮助。一些看起来完
美却无厘头的证明——除了线性代数,其他很多代数共同的
一点——是来自于一些在这些代数大厦建立以前的计算技巧
的总结。但是大多数却并未再被提及。

% In this book, the main goal is to answer all the
% questions about those are not well pondered in the
% traditional linear algebra course of any kind.
% In the computational matrix theory there are plenty
% ways of decompose the matrix, such as LU, QR, and the
% most fundamental SVD decomposition.
% While in the more math oriented linear course
% the main topic are always in the most abstract way
% as they move forward.
% And the main concepts such as the Transpose, dule
% space, determinant, trace and all other fundamental
% but always treated just in a forced memorize level.

% The main goal is to explain all the concepts in a 
% more mathematical natrual and fun way, even the sudden
% expose of a certain defination would be a very hard
% problem to those first expose to the abstract algebra
% world student just like me. So I'll try my best to
% get every defination and notion in a smooth and more
% reasonalbe way. 
}

\newcommand{\fakesections}{
\fakecontent{}
\section{Tensor oriented for computation}\fakecontent{}
\subsection{Examples}\fakecontent{}
\subsection{Excessive Elaborations}\fakecontent{}
\subsection{Long Winded Conclusion}\fakecontent{}
\subsection{Exercises}\fakecontent{}
\section{Theory vs Practice}\fakecontent{}
\subsection{Examples}\fakecontent{}
\subsection{Excessive Elaborations}\fakecontent{}
\subsection{Long Winded Conclusion}\fakecontent{}
\subsection{Exercises}\fakecontent{}
\foreach \x in {A,B,C,D,E,F,G,H,I,J,K,L,M,N,O,P,Q,R,S,T,U,V,W,X,Y,Z}
    {\index{\x 1}\index{\x 2}}
}

% Content Starts Here
\begin{document}
\begin{CJK}{UTF8}{gbsn}
    \frontmatter

    % ---- Half Title Page ----
    % current geometry will be restored after title page
    \newgeometry{top=1.75in,bottom=.5in}
    \begin{titlepage}
        \begin{flushleft}

            % Title
            \textbf{\fontfamily{qcs}\fontsize{48}{54}\selectfont 计线\\}

            % Draw a line 4pt high
            \par\noindent\rule{\textwidth}{4pt}\\

            % Subtitle
            % Shaded box from left to right. Text node is midway (centered).
            \begin{tikzpicture}
                \shade[bottom color=lightgray,top color=white]
                (0,0) rectangle (\textwidth, 1.5)
                node[midway] {\textbf{\large {\booksubtitle}}};
            \end{tikzpicture}

            % Edition Number
            \begin{flushright}
                \Large 第一版
            \end{flushright}

            % \vspace{\fill}
            \vspace{\fill}

        \end{flushleft}
    \end{titlepage}
    \restoregeometry
    % ---- End of Half Title Page ----


    % No page numbers on the Frontispiece page
    \thispagestyle{empty}


    % ---- Title Page ----
    % current geometry will be restored after title page
    \newgeometry{top=1.75in,bottom=.5in}
    \begin{titlepage}
        \begin{flushleft}

            % Title
            \textbf{\fontfamily{qcs}\fontsize{48}{54}\selectfont 计线\\}

            % Draw a line 4pt high
            \par\noindent\rule{\textwidth}{4pt}\\

            % Shaded box from left to right with Subtitle
            % The text node is midway (centered).
            \begin{tikzpicture}
                \shade[bottom color=lightgray,top color=white]
                (0,0) rectangle (\textwidth, 1.5)
                node[midway] {\textbf{\large {\booksubtitle}}};
            \end{tikzpicture}

            % Edition Number
            \begin{flushright}
                \Large 第一版
            \end{flushright}

            \vspace{\fill}

            % Author and Location
            \textbf{\large \bookauthor}\\[3.5pt]
            \textbf{\large {\authorsubtitle}}

            \vspace{\fill}

            % Self Publishing Logo. Free to use: CC0 license.
            % The source file is book.svg. If you change the svg, you must then convert
            % it to pdf. There are many online and offline tools available to do that.
            \begin{center}
                \includegraphics{booksvg.pdf}\\[4pt]
                \fontfamily{lmtt}\small{Self Publishers Worldwide\\
                    Seattle San Francisco New York\\
                    London Paris Rome Beijing Barcelona}
            \end{center}

        \end{flushleft}
    \end{titlepage}
    \restoregeometry
    % ---- End of Title Page ----

    % Do not show page numbers on colophon page
    \thispagestyle{empty}

    \begin{flushleft}
        \vspace*{\fill}
        This book was typeset using \LaTeX{} software.\\
        本书给那些爱我与恨我的人。\\
        \vspace{\fill}
        Copyright \textcopyright{} \the\year{}  \bookauthor\\
        License: \booklicense
    \end{flushleft}

    % A title page resets the page # to 1, but the second title page
    % was actually page 3. So add two to page counter.
    \addtocounter{page}{2}

    % The asterisk excludes chapter from the table of contents.
    \chapter{前言}
    \prefacePara{}
    % \[
    %     \left.
    %     \begin{aligned}
    %          & \sqrt{{dX_{1}}^{2} + {dX_{2}}^{2} + {dX_{3}}^{2}} \\
    %          & \qquad
    %         \begin{aligned}
    %                & = \left(1 + \frac{\kappa}{8\pi} \int \frac{\sigma\, dV_{0}}{r}\right)
    %             \sqrt{{dx_{1}}^{2} + {dx_{2}}^{2} + {dx_{3}}^{2}},                             \\
    %             dT & = \left(1 - \frac{\kappa}{8\pi} \int \frac{\sigma\, dV_{0}}{r}\right) dl.
    %         \end{aligned}
    %     \end{aligned}
    %     \right\}
    % \]

    % Three-level Table of Contents
    \setcounter{tocdepth}{3}
    \renewcommand{\contentsname}{目录}
    \tableofcontents

    \mainmatter

    \chapter{0 杂谈}

    % 每次阉牛我都在场。对于一般的公牛,只用刀割去即可。但是对于格外生性者,就须采取槌骟术,也就是割开阴囊,掏出睾丸,一木槌砸个稀烂。从此后手术者只知道吃草干活,别的什么都不知道,连杀都不用捆。

    % 后来我才知道生活就是一个缓慢受锤的过程,人一天天老下去,奢望也一天一天消失,之后变得像挨了槌的牛一样。
    % ——黄金时代

    数学并不能防止你受锤,但是在你的黄金时代
    认真地学习数学能防止你睡着,让你醒着受锤,清醒地感受疼痛,和绝望。但同时
    也给予你正确认识和描述世界的语言与改变世界的工具,希望你能用你手里的
    利剑斩开你的前路,希望创造,毁灭的力量就在你的手里了。——沃兹基硕得

    本章并不会证明地讲解若干中学代数之后可能继续发展代数的方向。

    这里面主要还是以坐标计算为主,希望开发若干和高维空间解析几何和物理
    有密切关系的代数运算方法。主要以计算方法为主,并不会牵涉
    公理化的进世代代数方法。

    \section{张量计算方法}

    \subsection{(a,b)-张量 \texorpdfstring{$\forall a,b \in \mathbb{N}$}{∀a,b∈N}的例子}

    $\left[\begin{matrix}
                1 & 2 \\
                3 & 4
            \end{matrix}\right]$
    是一个(1,1)-张量,有一个协变部分和一个逆变部分,
    这种张量其实就是矩阵。矩阵是线性变换的表示。

    $\left[\begin{matrix}
                1 \\
                3
            \end{matrix}\right]$
    是一个(0,1)-张量,这种张量有一个逆变部分,也就是
    之后线性代数中的列向量。

    $\left[\begin{matrix}
                1 & 3
            \end{matrix}\right]$
    是一个(1,0)-张量,这种张量有一个协变部分,也是之后
    线代中的协向量,对偶向量,行向量,线性泛函。

    在本书其余部分中,我会使用(a,b)-张量,这种jihao

    \subsection{(0,1)-张量 列向量}
    列向量没什么好说的,你写一个大括号,中间放一列东西就行,也许不是数也行。
    $\left[\begin{matrix}
                a \\
                b \\
                c
            \end{matrix}\right]$

    这样的向量就是列向量。

    这样的一列数是没有意义的。聚个例子,我有三个苹果一个
    桃子可以记作
    $\left[\begin{matrix}
                3 \\
                1
            \end{matrix}\right]$
    同时我要是有三个房子一个汽车,我也可以记成
    $\left[\begin{matrix}
                3 \\
                1
            \end{matrix}\right]$
    要是没有单位或者说“基底”来确定每一个分量代表的意义
    这个一列数就没啥意义(或者至少在纯粹的数学以外,很难
    找到意义)

    所以这两个
    $\left[\begin{matrix}
                3 \\
                1
            \end{matrix}\right]$
    应该写成一些基向量的多少倍的和的形式。
    比如把苹果作为基底$\vec{e_1}$,桃子作为基底$\vec{e_2}$,
    房子记作基底$\vec{\tilde{e_1}}$,汽车记作基底$\vec{\tilde{e_2}}$。
    对第一种情况就是$3\vec{e_1}+1\vec{e_2}$
    对第二种情况就是$3\vec{\tilde{e_1}}+1\vec{\tilde{e_2}}$
    我们先把这个东西记成一个更简洁的形式。
    第一种情况
    $\left[\begin{matrix}
                \vec{e_1} & \vec{e_2}
            \end{matrix}\right]
        \left[\begin{matrix}
                3 \\
                1
            \end{matrix}\right]=3\vec{e_1}+1\vec{e_2}$
    第二种情况
    $\left[\begin{matrix}
                \vec{\tilde{e_1}} & \vec{\tilde{e_2}}
            \end{matrix}\right]
        \left[\begin{matrix}
                3 \\
                1
            \end{matrix}\right]=3\vec{\tilde{e_1}}+1\vec{\tilde{e_2}}$
    这两个列向量的分量一样,但是因为基底不同可以有
    不同的意义。


    textbook in linear algebra is very abundant,
    include the linear operator oriented approach,
    which is the currently best one I have ever seen.

    This book will base on a computational oriented
    approach to the subject of the so called linear
    algebra.Linear algebra is such a vast topic that
    I'm having the most firghtened heart to output
    what I have learnt so far in this amazing land.

    In this book, the main goal is to answer all the
    questions about those are not well pondered in the
    traditional linear algebra course of any kind.
    In the computational matrix theory there are plenty
    ways of decompose the matrix, such as LU, QR, and the
    most fundamental SVD decomposition.
    While in the more math oriented linear course
    the main topic are always in the most abstract way
    as they move forward.
    And the main concepts such as the Transpose, dule
    space, determinant, trace and all other fundamental
    but always treated just in a forced memorize level.

    The main goal is to explain all the concepts in a
    more mathematical natrual and fun way, even the sudden
    expose of a certain defination would be a very hard
    problem to those first expose to the abstract algebra
    world student just like me. So I'll try my best to
    get every defination and notion in a smooth and more
    reasonalbe way.
    \subsection{(1,0)-张量 行向量}
    \subsection{(1,1)-张量 线性变换}
    \subsection{(2,0)-张量 双线性函数}
    \subsection{练习}
    \section{外代数计算方法}
    \subsection{几何代数举例}
    \subsection{二向量和三向量}
    \subsection{点积楔积和三维向量场的通量}
    \subsection{练习}
    \section{多项式和求根公式}
    \subsection{五次多项式没有根式解}
    任何一个关心数学,爱好数学的人,在这个新媒体泛滥
    的现今,一定都了解过一个网红数学定理。那就是
    “五次及其以上的代数方程没有求根公式”。要想完全证
    明这个定理,数学中的代数学进入了以抽象代数为代表
    的新的时代,进入了对一切数学结构统一的历程。本书
    并不会过多谈及这些高级的代数知识,如群论,Galois理
    论等。因为在这些全部的璀璨的群星之前,还有一些漫
    长的黑夜中缓慢探索的历史。简单来说,先看一下五次
    以下的代数方程究竟怎么用根式写出他的所有解其实是
    很有必要的。
    \subsection{直到四次方程的求根公式}
    一次方程和二次方程的求根公式对于本书的读者来说应
    该非常熟悉,我们先写出来。
    对一次方程$a_1x+a_0=0$的解是$a_1^{-1}(-a_0)$
    或者用更熟悉的方法写$\frac{-a_0}{a_1}$
    对二次方程$a_2x^2+a_1x+a_0=0$
    我们可以先完全平方化,然后开二次根。
    $(\sqrt[2]{a_2}x)^2+2\sqrt[2]{a_2}\frac{a_1}{2\sqrt[2]{a_2}}x+(\frac{a_1}{2\sqrt[2]{a_2}})^2=(\frac{a_1}{2\sqrt[2]{a_2}})^2-a_0$
    即$(x+\frac{a_1}{2a_2})^2=\frac{a_1^2-4a_2a_0}{4a_2^2}$
    开根号就转化成了一次方程。
    $x=\frac{-a_1+\sqrt[2]{a_1^2-4a_2a_0}}{2a_2}$

    到第三次方程,事情就不这么好办了。
    我们需要一个更加通用的思路。在历史上,因为人们会解的
    只有一次和二次,这个归纳样本量实在是太小了,我们可以
    有很多的思路来解释这得到这两个公式的过程。我们需要找
    到一个统一的思路,对三次方程用对二次方程一样的思路这
    样才能向更高的次数进发。其中一种是完全方化,二次的时
    候我们用了完全平方化,所有三次我们希望完全立方化。但
    是如果你拿着被先贤们凑出来的三次求根公式事后诸葛亮,
    你会发现这样行不通,因为求根公式里面有二次根号。

    现在事后诸葛亮的介绍在群论发展以后才得以统一的思路。
    现在详细介绍这个思路。
    \subsection{练习}
    \foreach \x in {A,B,C,D,E,F,G,H,I,J,K,L,M,N,O,P,Q,R,S,T,U,V,W,X,Y,Z}
        {\index{\x 1}\index{\x 2}}

    \chapter{1 张量(向量) 空间}

    就像「0 杂谈」中说过的那样,单独
    写一个堆有一定结构的分量组成各种
    类型的张量并没有什么意义,除非我
    们在某种基底的意义下讨论个张量。


    \section{张量基础}
    \subsection{基底, 坐标分量, 协变与逆变}
    \subsection{练习}
    \foreach \x in {A,B,C,D,E,F,G,H,I,J,K,L,M,N,O,P,Q,R,S,T,U,V,W,X,Y,Z}
        {\index{\x 1}\index{\x 2}}

    \chapter{2 使用数组表示}

    就像「0 杂谈」中说过的那样,单独
    写一个堆有一定结构的分量组成各种
    类型的张量并没有什么意义,除非我
    们在某种基底的意义下讨论个张量。

    \section{矩阵是列向量组成的行向量}
    \subsection{一行列向量}
    $\left[\begin{matrix}
                1 & 2 \\
                3 & 4
            \end{matrix}\right]$
    正常的线性代数课都会告诉你矩阵是这么写的。
    但是其实你可以有两种新的写法。
    $\left[\left[\begin{matrix}
                    1 \\
                    3
                \end{matrix}\right]
            \left[\begin{matrix}
                    2 \\
                    4
                \end{matrix}\right]\right]$
    \subsection{练习}
    \foreach \x in {A,B,C,D,E,F,G,H,I,J,K,L,M,N,O,P,Q,R,S,T,U,V,W,X,Y,Z}
        {\index{\x 1}\index{\x 2}}

    \chapter{3 度量与度规张量}

    \chapter{Re:1 向量空间}

    \chapter{Re:2 线性映射与作为表示的矩阵}

    \chapter{Re:3 内(外)积空间}

    这一章我们将会回答那些我曾经非常希望知道答案的
    问题,关于行列式,迹,转置,对偶与各种算子(矩
    阵)分解的相关结论和性质。我希望当看完这本书全
    书,和之后这一章后读者能够有一种近乎直观的感受
    感受到这些非常繁琐繁多的性质其实可以非常显然与
    直观。

    \section{内积,作为一种双线性函数}
    \subsection{内积,带度量的空间}

    $\left[\begin{matrix}
                1 & 2 \\
                3 & 4
            \end{matrix}\right]$
    是一个(1,1)-张量,有一个协变部分和一个逆变部分,
    这种张量其实就是矩阵。矩阵是线性变换的表示。

    $\left[\begin{matrix}
                1 \\
                3
            \end{matrix}\right]$
    是一个(0,1)-张量,这种张量有一个逆变部分,也就是
    之后线性代数中的列向量。

    $\left[\begin{matrix}
                1 & 3
            \end{matrix}\right]$
    是一个(1,0)-张量,这种张量有一个协变部分,也是之后
    线代中的协向量,对偶向量,行向量,线性泛函。

    \subsection{内积作为度量}
    列向量没什么好说的,你写一个大括号,中间放一列东西就行,也许不是数也行。
    $\left[\begin{matrix}
                a \\
                b \\
                c
            \end{matrix}\right]$

    这样的向量就是列向量。

    这样的一列数是没有意义的。聚个例子,我有三个苹果一个
    桃子可以记作
    $\left[\begin{matrix}
                3 \\
                1
            \end{matrix}\right]$
    同时我要是有三个房子一个汽车,我也可以记成
    $\left[\begin{matrix}
                3 \\
                1
            \end{matrix}\right]$
    要是没有单位或者说“基底”来确定每一个分量代表的意义
    这个一列数就没啥意义(或者至少在纯粹的数学以外,很难
    找到意义)

    所以这两个
    $\left[\begin{matrix}
                3 \\
                1
            \end{matrix}\right]$
    应该写成一些基向量的多少倍的和的形式。
    比如把苹果作为基底$\vec{e_1}$,桃子作为基底$\vec{e_2}$,
    房子记作基底$\vec{\tilde{e_1}}$,汽车记作基底$\vec{\tilde{e_2}}$。
    对第一种情况就是$3\vec{e_1}+1\vec{e_2}$
    对第二种情况就是$3\vec{\tilde{e_1}}+1\vec{\tilde{e_2}}$
    我们先把这个东西记成一个更简洁的形式。
    第一种情况
    $\left[\begin{matrix}
                \vec{e_1} & \vec{e_2}
            \end{matrix}\right]
        \left[\begin{matrix}
                3 \\
                1
            \end{matrix}\right]=3\vec{e_1}+1\vec{e_2}$
    第二种情况
    $\left[\begin{matrix}
                \vec{\tilde{e_1}} & \vec{\tilde{e_2}}
            \end{matrix}\right]
        \left[\begin{matrix}
                3 \\
                1
            \end{matrix}\right]=3\vec{\tilde{e_1}}+1\vec{\tilde{e_2}}$
    这两个列向量的分量一样,但是因为基底不同可以有
    不同的意义。


    textbook in linear algebra is very abundant,
    include the linear operator oriented approach,
    which is the currently best one I have ever seen.

    This book will base on a computational oriented
    approach to the subject of the so called linear
    algebra.Linear algebra is such a vast topic that
    I'm having the most firghtened heart to output
    what I have learnt so far in this amazing land.

    In this book, the main goal is to answer all the
    questions about those are not well pondered in the
    traditional linear algebra course of any kind.
    In the computational matrix theory there are plenty
    ways of decompose the matrix, such as LU, QR, and the
    most fundamental SVD decomposition.
    While in the more math oriented linear course
    the main topic are always in the most abstract way
    as they move forward.
    And the main concepts such as the Transpose, dule
    space, determinant, trace and all other fundamental
    but always treated just in a forced memorize level.

    The main goal is to explain all the concepts in a
    more mathematical natrual and fun way, even the sudden
    expose of a certain defination would be a very hard
    problem to those first expose to the abstract algebra
    world student just like me. So I'll try my best to
    get every defination and notion in a smooth and more
    reasonalbe way.
    \subsection{练习}
    \section{外积作为度量的可能性}
    \subsection{几何代数举例}
    \subsection{二向量和三向量}
    \subsection{点积楔积和三维向量场的通量}
    \subsection{练习}
    \foreach \x in {A,B,C,D,E,F,G,H,I,J,K,L,M,N,O,P,Q,R,S,T,U,V,W,X,Y,Z}
        {\index{\x 1}\index{\x 2}}

    % \chapter{Determinants}
    % \chapter{Quantics}
    % \chapter{Calculus}
    % \chapter{Differential Equations}
    % \chapter{Infinite Series}
    % \chapter{Theory of Functions}
    % \chapter{Probabilities and Least Squares}
    % \chapter{Analytic Geometry}
    % \chapter{Modern Geometry}
    % \chapter{Trigonometry and Elementary Geometry}
    % \chapter{Non-Euclidean Geometry}
    % \chapter{Bibliography}
    % \chapter{General Tendencies}
    \bibliographystyle{unsrt}
    \bibliography{std_book_cn_ref}


    \backmatter
    \addcontentsline{toc}{chapter}{Index}
    \printindex

\end{CJK}
\end{document}
